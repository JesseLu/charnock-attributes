openly publish it, though he dares secretly think it. He cannot raze
out the thoughts of a Deity, though he endeavors to blot those 
characters of God in his soul. He hath some doubts whether there be a
God or no: he wishes there were not any, and sometimes hopes there
is none at all. He could not so ascertain himself convincing
arguments to produce to the world, but he tampered with his own
heart to bring it to that persuasion, and smothered in himself those
notices of a Deity; which is so plain against the light of nature, that
such a man may well be called a fool for it.

\emph{There is no God}\footnote{HEBREW ``No God.'' Muis.} HEBREW 
\emph{non potestas Domini}, Chaldae. It is
not Jehovah, which name signifies the essence of God, as the prime
and supreme being; but Eloahia, which name signifies the providence
of God, God as a rector and judge. Not that he denies the existence 
of a Supreme Being, that created the world, but his regarding
the creatures, his government of the world, and consequently his
reward of the righteous or punishments of the wicked.

There is a threefold denial of God,\footnote{Cocceius} 1. \emph{Quoad existentiam}; this is
absolute atheism. 2. \emph{Quoad Providentiam}, or his inspection into, or
care of the things of the world, bounding him in the heavens. 3.
\emph{Quoad naturam}, in regard of one or other of the perfections due to
his nature. 

Of the denial of the providence of God most understand this, not
excluding the absolute atheist, as Diagoras is reported to be, nor the
skeptical atheist, as Protagoras, who doubted whether there were a
God.\footnote{Not owning him as the Egyptians called, GREEK. Eugubin in cloc.}
Those that deny the providence of God, do in effect deny the
being of God; for they strip him of that wisdom, goodness, tenderness,
mercy, justice, righteousness, which are the glory of the Deity.
And that principle, of a greedy desire to be uncontrolled in their
lusts, which induceth men to a denial of Providence, that thereby
they might stifle those seeds of fear which infect and embitter
their sinful pleasures, may as well lead them to deny that there
is any such being as a God. That at one blow, their fears may be 
dashed all in pieces and dissolved by the removal of the foundation:
as men who desire liberty to commit works of darkness, would not
have the lights in the house dimmed, but extinguished. What men
say against Providence, because they would have no check in their
lusts, they may say in their hearts against the existence of God upon
the same account; little difference between the dissenting from the
one and disowning the other.

\emph{They are corrupt, they have done abominable works, there is none that
doeth good.} He speaks of the atheist in the singular, ``the fool;'' of
the corruption issuing in the life in the plural; intimating that though
some few may choke in their hearts the sentiments of God and his
providence, and positively deny them, yet there is something of a
secret atheism in all, which is the fountain of the evil practices in
their lives, not an utter disowning of the being of a God, but a denial
or doubting of some of the rights of his nature. When men deny
the God of purity, they must needs be polluted in soul and body,
and grow brutish in their actions. When the sense of religion is

