\documentclass[a5paper]{book}
\usepackage[a5paper]{geometry}
\usepackage[utf8x]{inputenc}
\usepackage[hebrew,greek,english]{babel}
\newcommand{\heb}{[hebrew]}
\newcommand{\grk}{[greek]}

\title{Discourse 1: On the existence of God}
\date{Psalm 14:1---The fool hath said in his heart, There is no God. They are corrupt, they have done abominable works, there is none that doeth good.}
\begin{document}
\maketitle
\chapter{On the existence of God}
% Page 34.
\section{Exposition of Psalm 14:1}
This psalm is a description 
    of the deplorable corruption by nature of every son of Adam, 
    since the withering of that common root.
Some restrain it to the Gentiles, 
    as a wilderness full of briers and thorns, 
    as not concerning the Jews, the garden of God, 
    planted by his grace, and watered by the dew of heaven. 
But the apostle, the best interpreter, 
    rectifies this in extending it by name to Jews, as well as Gentiles, 
    (Rom. 3:9) ``We have before proved both Jews and Gentiles, 
    that they are all under sin;'' and 
    (ver. 10-12) cites part of this psalm and other passages of scripture 
    for the further evidence of it, 
    concluding by Jews and Gentiles, 
    every person in the world naturally in this state of corruption.

The psalmist first declares the corruption of the faculties of the soul, 
    \emph{The fool hath said in his heart}; 
    secondly, the streams issuing from thence, \emph{they are corrupt}, \&c.: 
    the first in atheistical principles,
    the other in unworthy practice; 
    and lays all the evil, tyranny, lust, and persecutions by men, 
    (as if the world were only for their sake) upon the neglects of God, 
    and the atheism cherished in their hearts.

\emph{The fool}, a term in scripture signifying a wicked man, 
    used also by the heathen philosophers to signify a vicious person, 
    \R{נבל} as coming from \R{נבל}? 
    signifies the extinction of life in men, animals, and plants; 
    so the word \R{נבל} is taken, 
    a plant that hath lost all that juice that made it lovely and useful.\footnote{
        Isaiah 40:7. \R{ציץ נבל} ``the flower fadeth.'', Isaiah 28:1} 
So a fool is one that hath lost his wisdom, 
    and right notion of God and divine things 
    which were communicated to man by creation; 
    one dead in sin, 
    yet one not so much void of rational faculties 
    as of grace in those faculties, 
    not one that wants reason, but abuses his reason. 
In Scripture the word signifies foolish.\footnote{
        Mais \R{נבל} and \R{חכם}(?)put together. 
        Deut. 32:6 ``O foolish people and unwise.''} 

\emph{Said in his heart}; that is, he thinks, or he doubts, or he wishes.
The thoughts of the heart are in the nature of words to God, though not to men. 
It is used in the like case of the atheistical person, 
    (Ps. 10:11, 13) ``He hath said in his heart, God hath forgotten; 
    he hath said in his heart, Thou wilt not require it.'' 
He doth not form a syllogism, as Calvin speaks, that there is no God: 
    he dares not % Page 35.
    openly publish it, though he dares secretly think it. 
He cannot raze out the thoughts of a Deity, 
    though he endeavors to blot those characters of God in his soul. 
He hath some doubts whether there be a God or no: 
    he wishes there were not any, 
    and sometimes hopes there is none at all. 
He could not so ascertain himself convincing arguments to produce to the world, 
    but he tampered with his own heart to bring it to that persuasion, 
    and smothered in himself those notices of a Deity; 
    which is so plain against the light of nature, 
    that such a man may well be called a fool for it.

\emph{There is no God}\footnote{\heb ``No God.'' Muis.} 
    \heb \emph{non potestas Domini}, Chaldae. 
It is not Jehovah, which name signifies the essence of God, 
    as the prime and supreme being; 
    but Eloahia, which name signifies the providence of God, 
    God as a rector and judge. 
Not that he denies the existence of a Supreme Being, 
    that created the world, but his regarding the creatures, 
    his government of the world, 
    and consequently his reward of the righteous or punishments of the wicked.

There is a threefold denial of God,\footnote{Cocceius} 
    1. \emph{Quoad existentiam}; this is absolute atheism. 
    2. \emph{Quoad Providentiam}, or his inspection into, 
    or care of the things of the world, bounding him in the heavens. 
    3. \emph{Quoad naturam}, in regard of one or other of the perfections 
    due to his nature. 

Of the denial of the providence of God most understand this, 
    not excluding the absolute atheist, as Diagoras is reported to be, 
    nor the skeptical atheist, as Protagoras, 
    who doubted whether there were a God.\footnote{
    Not owning him as the Egyptians called, \grk . Eugubin in cloc.}
Those that deny the providence of God, do in effect deny the being of God; 
    for they strip him of that wisdom, goodness, tenderness,
    mercy, justice, righteousness, which are the glory of the Deity.
And that principle, of a greedy desire to be uncontrolled in their lusts, 
    which induceth men to a denial of Providence, 
    that thereby they might stifle those seeds of fear 
    which infect and embitter their sinful pleasures, 
    may as well lead them to deny that there is any such being as a God. 
That at one blow, their fears may be dashed all in pieces 
    and dissolved by the removal of the foundation:
    as men who desire liberty to commit works of darkness, 
    would not have the lights in the house dimmed, but extinguished. 
What men say against Providence, 
    because they would have no check in their lusts, 
    they may say in their hearts against the existence of God 
    upon the same account; 
    little difference between the dissenting from the one and 
    disowning the other.

\emph{They are corrupt, they have done abominable works, 
    there is none that doeth good.} 
He speaks of the atheist in the singular, ``the fool;'' 
    of the corruption issuing in the life in the plural; 
    intimating that though some few may choke in their hearts 
    the sentiments of God and his providence, and positively deny them, 
    yet there is something of a secret atheism in all, 
    which is the fountain of the evil practices in their lives, 
    not an utter disowning of the being of a God, 
    but a denial or doubting of some of the rights of his nature. 
When men deny the God of purity, 
    they must needs be polluted in soul and body,
    and grow brutish in their actions. 
When the sense of religion is % Page 36.
    shaken off, 
    all kinds of wickedness is eagerly rushed into, 
    whereby they become as loathsome to God as putrefied carcasses are to men.\footnote{
    Atheism absolute is not in all men's judgements,
    but practial is in all men's actions.
    The Apostle in the Romans applying the latter part of it to all mankind,
    but not the former;
    as the word translated \emph{corrupt} signifies.}
Not one or two evil actions is the product of such a principle, 
    but the whole scene of a man’s life is corrupted and becomes execrable.

No man is exempted from some spice of atheism by the depravation of his nature,
    which the psalmist intimates, 
    ``there is none that doeth good:''
    though there are indelible convictions of the being of a God,
    that they cannot absolutely deny it; 
    yet there are some atheistical bubblings in the hearts of men, 
    which evidence themselves in their actions. 
As the apostle, (Titus 1:16) ``They profess that they know God, 
    but in works the deny him.'' 
Evil works are a dust stirred up by an atheistical breath. 
He that habituates himself in some sordid lust, 
    can scarcely be said seriously and firmly to believe 
    that there is a God in being; 
    and the apostle doth not say that they know God, 
    but they profess to know him: 
    true knowledge and profession of knowledge are distinct.
It intimates also to us, 
    the unreasonableness of atheism in the consequence, 
    when men shut their eyes against the beams of so clear a sun,
    God revengeth himself upon for their impiety, 
    by leaving them to their own wills, 
    lets them fall into the deepest sink and dregs of iniquity; 
    and since they doubt of him in their hearts, 
    suffers them above others to deny him in their works, 
    this the apostle discourscth at large.\footnote{Romans 1:24}
The text then is a description of man’s corruption.

1. Of his mind. \emph{The fool hath said in his heart}. 
    No better title than that of a fool is afforded to the atheist.

2. Of the other faculties, 
    1. In sins of commission, expressed by the loathsomeness
    (\emph{corrupt}. \emph{abominable}); 
    2. In sins of omission (\emph{there is none that doeth good})
    he lays down the corruption of the mind as the cause,
    the corruption of the other faculties as the effect.
% This seems like the end of the exposition of Psalm 14:1.

\section{Outline}
I. It is a great folly to deny or doubt of the existence or being of God:
    or, the atheist is a great fool.

II. Practical atheism is natural to man in his corrupt state. 
It is against nature as constituted by God, but natural, 
    as nature is depraved by man: 
    the absolute disowning of the being of a God is not natural to men, 
    but the contrary is natural; 
    but an inconsideration of God, or misrepresentation of his nature, 
    is natural to man as corrupt.

III. A secret atheism, or a partial atheism, 
    is the spring of all the wicked practices in the world: 
    the disorders of the life spring from the ill dispositions of the heart.

\subsection{Every atheist is a grand fool}
For the first, every atheist is a grand fool. 
If he were not a fool, he would not imagine a thing so contrary 
    to the stream of the universal reason of the world, 
    contrary to the rational dictates of his own soul, 
    and contrary to the testimony of every creature, 
    and link in the chain of creation: 
    if he were not a fool, he would not strip himself of humanity, 
    and degrade himself lower than the most despicable brute. 
It is a folly; for though God be so inaccessible 
    that we cannot know him perfectly, 
    yet he is so much in the light, 
    that we cannot be totally ignorant of him; 
    as he cannot be comprehended in his essence, 
    he cannot be unknown in his existence; 
    it is as easy by reason to understand that he is, 
    as it is difficult to know what he is. 
The demonstrations reason furnisheth us with for the existence of God, 
    will be evidences of the atheist’s folly. 
One would think there were little need of spending time 
    in evidencing this truth, since in the principle of it, 
    it seems to be so universally owned, 
    and at the first proposal and demand, 
    gains the assent of most men.

But, 1. Doth not the growth of atheism among us render this necessary? 
    may it not justly be suspected, 
    that the swarms of atheists are more numerous in our times, 
    than history records to have been in any age, 
    when men will not only say it in their hearts, 
    but publish it with their lips, 
    and boast that they have shaken off those shackles 
    which bind other men’s consciences? 
Doth not the barefaced debauchery of men evidence such a settled sentiment, 
    or at least a careless belief of the truth, 
    which lies at the root, 
    and sprouts up in such venomous branches in the world? 
Can men’s hearts be free from that principle wherewith 
    their practices are so openly depraved? 
It is true, the light of nature shines too vigorously 
    for the power of man totally to put it out; 
    yet loathsome actions impair and weaken 
    the actual thoughts and considerations of a Deity, 
    and are like mists that darken the light of the sun, 
    though they cannot extinguish it: 
    their consciences, as a candlestick, must hold it, 
    though their unrighteousness obscure it, 
    (Rom. 1:18) ``Who hold the truth in unrighteousness.'' 
The engraved characters of the law of nature remain, 
    though they daub them with their muddy lusts to make them illegible: 
    so that since the inconsideration of a Deity 
    is the cause of all the wickedness and extravagances of men; 
    and as Austin saith,
    the proposition is always true, 
    the fool hath said in his heart, \&c.
    and more evidently true in this age than any, 
    it will not be unnecessary to discourse 
    of the demonstrations of this first principle. 
The apostles spent little time in urging this truth; 
    it was taken for granted all over the world, 
    and they were generally devout in the worship
    of those idols they thought to be gods: 
    that we run from one God to many, 
    and our age is running from one God to none at all.

2. The existence of God is the foundation of all religion. 
The whole building totters if the foundation be out of course: 
    if we have not deliberate and right notions of it, 
    we shall perform no worship, no service, yield no affection to him. 
If there be not a God, it is impossible there can be one, 
    for eternity is essential to the notion of a God; 
    so all religion would be vain, 
    and unreasonable to pay homage to that which is not in being, 
    nor can ever be. 
We must first believe that he is, 
    and that he is what he declares himself to be, 
    before we can seek him, adore him, and devote our affections to him.\footnote{Hebrews 11:6}
We cannot pay God a due and regular homage, 
    unless we understand him in is perfections, what he is; 
    and we can pay him no homage at all, unless we believe that he is.

3. It is fit we should know why we believe, 
    that our belief of a God may appear to be upon undeniable evidence, 
    and that we may give a better reason for his existence, 
    than that we have heard our % Begin page 0038.
    parents and teachers tell us so, and our acquaintance think so. 
It is as much as to say there is no God, 
    when we know not why we believe there is, 
    and would not consider the arguments for his existence.

4. It is necessary to depress that secret atheism which is in the
    heart of every man by nature. 
Though every visible object which offers itself to our sense, 
    presents a deity to our minds, 
    and exhorts us to subscribe to the truth of it; 
    yet there is a root of atheism springing sometimes 
    in wavering thoughts and foolish imaginations, 
    inordinate actions, and secret wishes. 
Certain it is, that every man that doth not love God, denies God; 
    now can he that disaffects him, and hath a slavish fear of him, 
    wish his existence, and say to his own heart with any cheerfulness, 
    there is a God, and make it his chief care to persuade himself of it? 
    he would persuade himself there is no God, 
    and stifle the seeds of it in his reason and conscience, 
    that he might have the greatest liberty to entertain 
    the allurements of the flesh. 
It is necessary to excite men to daily and actual considerations 
    of God and his nature, 
    which would be a bar to much of that wickedness which overflows 
    in the lives of men.

5. Nor is it unuseful to those who effectually believe and love him;\footnote{
        Coccei Sum. Theol. c. 8 s. 1}
    for those who have had a converse with God, 
    and felt his powerlful influences in the secrets of their hearts, 
    to take a prospect of those satisfactory accounts 
    which reason gives of that God they adore and love; 
    to see every creature justify them in their owning of him, 
    and affections to him: 
    indeed the evidences of a God striking upon the conscience 
    of those who resolve to cleave to sin as their chiefest darling, 
    will dash their pleasures with unwelcome mixtures.

% Seems to start new section.
I shall further premise this, 
    That the folly of atheism is evidenced by the light of reason. 
Men that will not listen to Scripture, 
    as having no counterpart of it in their souls, 
    cannot easily deny natural reason, 
    which rise up on all sides for the justiication of this truth.
There is a natural as well as a revealed knowledge, 
    and the book of the creatures is legible in declaring the being of a God, 
    as well as the Scriptures are in declaring the nature of a God; 
    there are outward objects in the world, 
    and common principles in the conscience, whence it may be inferred.

For, 1. God in regard of his existence is not only the discovery of faith, 
    but of reason. 
God hath revealed not only his being, 
    but some sparks of his eternal power and godhead in his works, 
    as well as in his word. 
(Rom. 1:19, 20), “God hath showed it unto them,”---how?\footnote{Aquin.}
    in his works; by the things that are made, 
    it is a discovery to our reason, as shining in the creatures; 
    and an object of our faith as breaking out upon us in the Scriptures: 
    it is an article of our faith, and an article of our reason. 
Faith supposeth natural knowledge, as grace supposeth nature. 
Faith indeed is properly of things above reason, 
    purely depending upon revelation. 
What can be demonstrated by natural light, 
    is not so properly the object of faith;
    though in regard of the addition of a certainty by revelation it is so.
The belief that God is, which the apostle speaks of,\footnote{Hebrews 11:6}
    is not so much of the bare existence of God, 
    as what God is in relation to them that
    seek him, viz. a rewarder.  %% Begin 0039
The apostle speaks of the faith of Abel,
    the faith of Enoch, 
    such a faith thiat pleases God: 
    but the faith of Abel testified in his sacriice, 
    and the faith of Enoch testified in his walking with God, 
    was not simply a faith of the existence of God.
Cain in the time of Abel, 
    other men in the world in the time of Enoch, 
    believed this as well as they: 
    but it was a faith joined with the worship of God, 
    and desires to please him in the way of his own appointment; 
    so that they believe that God was 
    such as he had declared himself to be in his promise to Adam, 
    such an one as would be as good as his word, 
    and bruise the serpent’s head. 
He that seeks to God according to the mind of God, 
    must believe that he is such a God that will pardon sin, 
    and justify a seeker of him; 
    that he is a God of that ability and will, 
    to justify a sinner in that way he hath appointed 
    for the clearing the holiness of his nature, 
    and vindicating the honor of his law violated by man. 
No man can seek God or love God, 
    unless he believe him to be thus; 
    and he cannot seek God without a discovery of his own mind 
    how he would be sought. 
For it is not a seeking God in any way of man's invention, 
    that renders him capable of this desired fruit of a reward. 
He that believes God as a rewarder, 
    must believe the promise of God concerning the Messiah. 
Men under the conscience of sin, 
    cannot tell without a divine discovery, 
    whether God will reward, 
    or how he will reward the seekers of him; 
    and therefore cannot act towards him as an object of faith. 
Would any man seek God merely because he is, 
    or love him because he is, 
    if he did not know that he should be acceptable to him?
The bare existence of a thing is not the ground of affection to it, 
    but those qualities of it and our interest in it, 
    which render it amiable and delightful. 
How can men, 
    whose consciences fly in their faces seek God or love him, 
    without this knowledge that he is a rewarder? 
Nature doth not show any way to a sinner, 
    how to reconcile God’s provoked justice with his tenderness. 
The faith the apostle speaks of here is 
    a faith that eyes the reward as an encouragement, 
    and the will of God as the rule of its acting; 
    he doth not speak simply of the existence of God.

I have spoken the more of this place, 
    because the Socinians\footnote{Voet. Theol. Natural. cap. 3. S. 1. p. 22.} 
    use this to decry any natural knowledge of God, 
    and that the existence of God is only to be known by revelation, 
    so that by that reason any one that lived without the Scripture 
    hath no ground to believe the being of a God. 
The Scripture ascribes a knowledge of God 
    to all nations in the world (Rom. 1:19); 
    not only a faculty of knowing, 
    if they had arguments and demonstrations, 
    as an ignorant man in any art hath a faculty to know; 
    but it ascribes an actual knowledge (v. 10) ``manifest in them;'' 
    (v. 21) ``They knew God;'' not they might know him; 
    they knew him when they did not care for knowing him. 
The notices of God are as intelligible to us by reason, 
    as any object in the world is visible; 
    he is written in every letter.

2. We are often in the Scripture sent 
    to take a prospect of the creatures for a discovery of God. 
The apostles drew arguments from the topics of nature, 
    when they discoursed with those that owned the Scripture (Rom. 1:19), 
    as well as when they treated with those


\end{document}

