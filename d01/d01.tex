\documentclass[a5paper]{book}
\usepackage[a5paper]{geometry}
\usepackage[utf8x]{inputenc}
\usepackage[hebrew,greek,english]{babel}
\newcommand{\heb}{[hebrew]}
\newcommand{\grk}{[greek]}

\title{Charnock on the Attributes}
% \date{Psalm 14:1----The fool hath said in his heart, There is no God. They are corrupt, they have done abominable works, there is none that doeth good.}
\begin{document}
\maketitle
\tableofcontents
\chapter{On the existence of God}
\begin{quote}Psalm 14:1----The fool hath said in his heart, There is no God. They are corrupt, they have done abominable works, there is none that doeth good.
\end{quote}
% Page 34.
\section{Introduction}
\subsection{Exposition of Psalm 14:1}
This psalm is a description 
    of the deplorable corruption by nature of every son of Adam, 
    since the withering of that common root.
Some restrain it to the Gentiles, 
    as a wilderness full of briers and thorns, 
    as not concerning the Jews, the garden of God, 
    planted by his grace, and watered by the dew of heaven. 
But the apostle, the best interpreter, 
    rectifies this in extending it by name to Jews, as well as Gentiles, 
    (Rom. 3:9) ``We have before proved both Jews and Gentiles, 
    that they are all under sin;'' and 
    (ver. 10-12) cites part of this psalm and other passages of scripture 
    for the further evidence of it, 
    concluding by Jews and Gentiles, 
    every person in the world naturally in this state of corruption.

The psalmist first declares the corruption of the faculties of the soul, 
    \emph{The fool hath said in his heart}; 
    secondly, the streams issuing from thence, \emph{they are corrupt}, \&c.: 
    the first in atheistical principles,
    the other in unworthy practice; 
    and lays all the evil, tyranny, lust, and persecutions by men, 
    (as if the world were only for their sake) upon the neglects of God, 
    and the atheism cherished in their hearts.

\emph{The fool}, a term in scripture signifying a wicked man, 
    used also by the heathen philosophers to signify a vicious person, 
    \R{נבל} as coming from \R{נבל}? 
    signifies the extinction of life in men, animals, and plants; 
    so the word \R{נבל} is taken, 
    a plant that hath lost all that juice that made it lovely and useful.\footnote{
        Isaiah 40:7. \R{ציץ נבל} ``the flower fadeth.'', Isaiah 28:1} 
So a fool is one that hath lost his wisdom, 
    and right notion of God and divine things 
    which were communicated to man by creation; 
    one dead in sin, 
    yet one not so much void of rational faculties 
    as of grace in those faculties, 
    not one that wants reason, but abuses his reason. 
In Scripture the word signifies foolish.\footnote{
        Mais \R{נבל} and \R{חכם}(?)put together. 
        Deut. 32:6 ``O foolish people and unwise.''} 

\emph{Said in his heart}; that is, he thinks, or he doubts, or he wishes.
The thoughts of the heart are in the nature of words to God, though not to men. 
It is used in the like case of the atheistical person, 
    (Ps. 10:11, 13) ``He hath said in his heart, God hath forgotten; 
    he hath said in his heart, Thou wilt not require it.'' 
He doth not form a syllogism, as Calvin speaks, that there is no God: 
    he dares not % Page 35.
    openly publish it, though he dares secretly think it. 
He cannot raze out the thoughts of a Deity, 
    though he endeavors to blot those characters of God in his soul. 
He hath some doubts whether there be a God or no: 
    he wishes there were not any, 
    and sometimes hopes there is none at all. 
He could not so ascertain himself convincing arguments to produce to the world, 
    but he tampered with his own heart to bring it to that persuasion, 
    and smothered in himself those notices of a Deity; 
    which is so plain against the light of nature, 
    that such a man may well be called a fool for it.

\emph{There is no God}\footnote{\heb ``No God.'' Muis.} 
    \heb \emph{non potestas Domini}, Chaldae. 
It is not Jehovah, which name signifies the essence of God, 
    as the prime and supreme being; 
    but Eloahia, which name signifies the providence of God, 
    God as a rector and judge. 
Not that he denies the existence of a Supreme Being, 
    that created the world, but his regarding the creatures, 
    his government of the world, 
    and consequently his reward of the righteous or punishments of the wicked.

There is a threefold denial of God,\footnote{Cocceius} 
    1. \emph{Quoad existentiam}; this is absolute atheism. 
    2. \emph{Quoad Providentiam}, or his inspection into, 
    or care of the things of the world, bounding him in the heavens. 
    3. \emph{Quoad naturam}, in regard of one or other of the perfections 
    due to his nature. 

Of the denial of the providence of God most understand this, 
    not excluding the absolute atheist, as Diagoras is reported to be, 
    nor the skeptical atheist, as Protagoras, 
    who doubted whether there were a God.\footnote{
    Not owning him as the Egyptians called, \grk . Eugubin in cloc.}
Those that deny the providence of God, do in effect deny the being of God; 
    for they strip him of that wisdom, goodness, tenderness,
    mercy, justice, righteousness, which are the glory of the Deity.
And that principle, of a greedy desire to be uncontrolled in their lusts, 
    which induceth men to a denial of Providence, 
    that thereby they might stifle those seeds of fear 
    which infect and embitter their sinful pleasures, 
    may as well lead them to deny that there is any such being as a God. 
That at one blow, their fears may be dashed all in pieces 
    and dissolved by the removal of the foundation:
    as men who desire liberty to commit works of darkness, 
    would not have the lights in the house dimmed, but extinguished. 
What men say against Providence, 
    because they would have no check in their lusts, 
    they may say in their hearts against the existence of God 
    upon the same account; 
    little difference between the dissenting from the one and 
    disowning the other.

\emph{They are corrupt, they have done abominable works, 
    there is none that doeth good.} 
He speaks of the atheist in the singular, ``the fool;'' 
    of the corruption issuing in the life in the plural; 
    intimating that though some few may choke in their hearts 
    the sentiments of God and his providence, and positively deny them, 
    yet there is something of a secret atheism in all, 
    which is the fountain of the evil practices in their lives, 
    not an utter disowning of the being of a God, 
    but a denial or doubting of some of the rights of his nature. 
When men deny the God of purity, 
    they must needs be polluted in soul and body,
    and grow brutish in their actions. 
When the sense of religion is % Page 36.
    shaken off, 
    all kinds of wickedness is eagerly rushed into, 
    whereby they become as loathsome to God as putrefied carcasses are to men.\footnote{
    Atheism absolute is not in all men's judgements,
    but practial is in all men's actions.
    The Apostle in the Romans applying the latter part of it to all mankind,
    but not the former;
    as the word translated \emph{corrupt} signifies.}
Not one or two evil actions is the product of such a principle, 
    but the whole scene of a man’s life is corrupted and becomes execrable.

No man is exempted from some spice of atheism by the depravation of his nature,
    which the psalmist intimates, 
    ``there is none that doeth good:''
    though there are indelible convictions of the being of a God,
    that they cannot absolutely deny it; 
    yet there are some atheistical bubblings in the hearts of men, 
    which evidence themselves in their actions. 
As the apostle, (Titus 1:16) ``They profess that they know God, 
    but in works the deny him.'' 
Evil works are a dust stirred up by an atheistical breath. 
He that habituates himself in some sordid lust, 
    can scarcely be said seriously and firmly to believe 
    that there is a God in being; 
    and the apostle doth not say that they know God, 
    but they profess to know him: 
    true knowledge and profession of knowledge are distinct.
It intimates also to us, 
    the unreasonableness of atheism in the consequence, 
    when men shut their eyes against the beams of so clear a sun,
    God revengeth himself upon for their impiety, 
    by leaving them to their own wills, 
    lets them fall into the deepest sink and dregs of iniquity; 
    and since they doubt of him in their hearts, 
    suffers them above others to deny him in their works, 
    this the apostle discourscth at large.\footnote{Romans 1:24}
The text then is a description of man’s corruption.

1. Of his mind. \emph{The fool hath said in his heart}. 
    No better title than that of a fool is afforded to the atheist.

2. Of the other faculties, 
    1. In sins of commission, expressed by the loathsomeness
    (\emph{corrupt}. \emph{abominable}); 
    2. In sins of omission (\emph{there is none that doeth good})
    he lays down the corruption of the mind as the cause,
    the corruption of the other faculties as the effect.
% This seems like the end of the exposition of Psalm 14:1.

% \subsection{Outline}
I. It is a great folly to deny or doubt of the existence or being of God:
    or, the atheist is a great fool.

II. Practical atheism is natural to man in his corrupt state. 
It is against nature as constituted by God, but natural, 
    as nature is depraved by man: 
    the absolute disowning of the being of a God is not natural to men, 
    but the contrary is natural; 
    but an inconsideration of God, or misrepresentation of his nature, 
    is natural to man as corrupt.

III. A secret atheism, or a partial atheism, 
    is the spring of all the wicked practices in the world: 
    the disorders of the life spring from the ill dispositions of the heart.

\subsection{Every atheist is a grand fool}
For the first, every atheist is a grand fool. 
If he were not a fool, he would not imagine a thing so contrary 
    to the stream of the universal reason of the world, 
    contrary to the rational dictates of his own soul, 
    and contrary to the testimony of every creature, 
    and link in the chain of creation: 
    if he were not a fool, he would not strip himself of humanity, 
    and degrade himself lower than the most despicable brute. 
It is a folly; for though God be so inaccessible 
    that we cannot know him perfectly, 
    yet he is so much in the light, 
    that we cannot be totally ignorant of him; 
    as he cannot be comprehended in his essence, 
    he cannot be unknown in his existence; 
    it is as easy by reason to understand that he is, 
    as it is difficult to know what he is. 
The demonstrations reason furnisheth us with for the existence of God, 
    will be evidences of the atheist’s folly. 
One would think there were little need of spending time 
    in evidencing this truth, since in the principle of it, 
    it seems to be so universally owned, 
    and at the first proposal and demand, 
    gains the assent of most men.

But, 1. Doth not the growth of atheism among us render this necessary? 
    may it not justly be suspected, 
    that the swarms of atheists are more numerous in our times, 
    than history records to have been in any age, 
    when men will not only say it in their hearts, 
    but publish it with their lips, 
    and boast that they have shaken off those shackles 
    which bind other men’s consciences? 
Doth not the barefaced debauchery of men evidence such a settled sentiment, 
    or at least a careless belief of the truth, 
    which lies at the root, 
    and sprouts up in such venomous branches in the world? 
Can men’s hearts be free from that principle wherewith 
    their practices are so openly depraved? 
It is true, the light of nature shines too vigorously 
    for the power of man totally to put it out; 
    yet loathsome actions impair and weaken 
    the actual thoughts and considerations of a Deity, 
    and are like mists that darken the light of the sun, 
    though they cannot extinguish it: 
    their consciences, as a candlestick, must hold it, 
    though their unrighteousness obscure it, 
    (Rom. 1:18) ``Who hold the truth in unrighteousness.'' 
The engraved characters of the law of nature remain, 
    though they daub them with their muddy lusts to make them illegible: 
    so that since the inconsideration of a Deity 
    is the cause of all the wickedness and extravagances of men; 
    and as Austin saith,
    the proposition is always true, 
    the fool hath said in his heart, \&c.
    and more evidently true in this age than any, 
    it will not be unnecessary to discourse 
    of the demonstrations of this first principle. 
The apostles spent little time in urging this truth; 
    it was taken for granted all over the world, 
    and they were generally devout in the worship
    of those idols they thought to be gods: 
    that we run from one God to many, 
    and our age is running from one God to none at all.

2. The existence of God is the foundation of all religion. 
The whole building totters if the foundation be out of course: 
    if we have not deliberate and right notions of it, 
    we shall perform no worship, no service, yield no affection to him. 
If there be not a God, it is impossible there can be one, 
    for eternity is essential to the notion of a God; 
    so all religion would be vain, 
    and unreasonable to pay homage to that which is not in being, 
    nor can ever be. 
We must first believe that he is, 
    and that he is what he declares himself to be, 
    before we can seek him, adore him, and devote our affections to him.\footnote{Hebrews 11:6}
We cannot pay God a due and regular homage, 
    unless we understand him in is perfections, what he is; 
    and we can pay him no homage at all, unless we believe that he is.

3. It is fit we should know why we believe, 
    that our belief of a God may appear to be upon undeniable evidence, 
    and that we may give a better reason for his existence, 
    than that we have heard our % Begin page 0038.
    parents and teachers tell us so, and our acquaintance think so. 
It is as much as to say there is no God, 
    when we know not why we believe there is, 
    and would not consider the arguments for his existence.

4. It is necessary to depress that secret atheism which is in the
    heart of every man by nature. 
Though every visible object which offers itself to our sense, 
    presents a deity to our minds, 
    and exhorts us to subscribe to the truth of it; 
    yet there is a root of atheism springing sometimes 
    in wavering thoughts and foolish imaginations, 
    inordinate actions, and secret wishes. 
Certain it is, that every man that doth not love God, denies God; 
    now can he that disaffects him, and hath a slavish fear of him, 
    wish his existence, and say to his own heart with any cheerfulness, 
    there is a God, and make it his chief care to persuade himself of it? 
    he would persuade himself there is no God, 
    and stifle the seeds of it in his reason and conscience, 
    that he might have the greatest liberty to entertain 
    the allurements of the flesh. 
It is necessary to excite men to daily and actual considerations 
    of God and his nature, 
    which would be a bar to much of that wickedness which overflows 
    in the lives of men.

5. Nor is it unuseful to those who effectually believe and love him;\footnote{
        Coccei Sum. Theol. c. 8 s. 1}
    for those who have had a converse with God, 
    and felt his powerlful influences in the secrets of their hearts, 
    to take a prospect of those satisfactory accounts 
    which reason gives of that God they adore and love; 
    to see every creature justify them in their owning of him, 
    and affections to him: 
    indeed the evidences of a God striking upon the conscience 
    of those who resolve to cleave to sin as their chiefest darling, 
    will dash their pleasures with unwelcome mixtures.

% Seems to start new section.
\subsection{God's existence is evidenced by natural reason}
I shall further premise this, 
    That the folly of atheism is evidenced by the light of reason. 
Men that will not listen to Scripture, 
    as having no counterpart of it in their souls, 
    cannot easily deny natural reason, 
    which rise up on all sides for the justiication of this truth.
There is a natural as well as a revealed knowledge, 
    and the book of the creatures is legible in declaring the being of a God, 
    as well as the Scriptures are in declaring the nature of a God; 
    there are outward objects in the world, 
    and common principles in the conscience, whence it may be inferred.

For, 1. God in regard of his existence is not only the discovery of faith, 
    but of reason. 
God hath revealed not only his being, 
    but some sparks of his eternal power and godhead in his works, 
    as well as in his word. 
(Rom. 1:19, 20), “God hath showed it unto them,”----how?\footnote{Aquin.}
    in his works; by the things that are made, 
    it is a discovery to our reason, as shining in the creatures; 
    and an object of our faith as breaking out upon us in the Scriptures: 
    it is an article of our faith, and an article of our reason. 
Faith supposeth natural knowledge, as grace supposeth nature. 
Faith indeed is properly of things above reason, 
    purely depending upon revelation. 
What can be demonstrated by natural light, 
    is not so properly the object of faith;
    though in regard of the addition of a certainty by revelation it is so.
The belief that God is, which the apostle speaks of,\footnote{Hebrews 11:6}
    is not so much of the bare existence of God, 
    as what God is in relation to them that
    seek him, viz. a rewarder.  %% Begin 0039
The apostle speaks of the faith of Abel,
    the faith of Enoch, 
    such a faith thiat pleases God: 
    but the faith of Abel testified in his sacriice, 
    and the faith of Enoch testified in his walking with God, 
    was not simply a faith of the existence of God.
Cain in the time of Abel, 
    other men in the world in the time of Enoch, 
    believed this as well as they: 
    but it was a faith joined with the worship of God, 
    and desires to please him in the way of his own appointment; 
    so that they believe that God was 
    such as he had declared himself to be in his promise to Adam, 
    such an one as would be as good as his word, 
    and bruise the serpent’s head. 
He that seeks to God according to the mind of God, 
    must believe that he is such a God that will pardon sin, 
    and justify a seeker of him; 
    that he is a God of that ability and will, 
    to justify a sinner in that way he hath appointed 
    for the clearing the holiness of his nature, 
    and vindicating the honor of his law violated by man. 
No man can seek God or love God, 
    unless he believe him to be thus; 
    and he cannot seek God without a discovery of his own mind 
    how he would be sought. 
For it is not a seeking God in any way of man's invention, 
    that renders him capable of this desired fruit of a reward. 
He that believes God as a rewarder, 
    must believe the promise of God concerning the Messiah. 
Men under the conscience of sin, 
    cannot tell without a divine discovery, 
    whether God will reward, 
    or how he will reward the seekers of him; 
    and therefore cannot act towards him as an object of faith. 
Would any man seek God merely because he is, 
    or love him because he is, 
    if he did not know that he should be acceptable to him?
The bare existence of a thing is not the ground of affection to it, 
    but those qualities of it and our interest in it, 
    which render it amiable and delightful. 
How can men, 
    whose consciences fly in their faces seek God or love him, 
    without this knowledge that he is a rewarder? 
Nature doth not show any way to a sinner, 
    how to reconcile God’s provoked justice with his tenderness. 
The faith the apostle speaks of here is 
    a faith that eyes the reward as an encouragement, 
    and the will of God as the rule of its acting; 
    he doth not speak simply of the existence of God.

I have spoken the more of this place, 
    because the Socinians\footnote{Voet. Theol. Natural. cap. 3. S. 1. p. 22.} 
    use this to decry any natural knowledge of God, 
    and that the existence of God is only to be known by revelation, 
    so that by that reason any one that lived without the Scripture 
    hath no ground to believe the being of a God. 
The Scripture ascribes a knowledge of God 
    to all nations in the world (Rom. 1:19); 
    not only a faculty of knowing, 
    if they had arguments and demonstrations, 
    as an ignorant man in any art hath a faculty to know; 
    but it ascribes an actual knowledge (v. 10) ``manifest in them;'' 
    (v. 21) ``They knew God;'' not they might know him; 
    they knew him when they did not care for knowing him. 
The notices of God are as intelligible to us by reason, 
    as any object in the world is visible; 
    he is written in every letter.

2. We are often in the Scripture sent 
    to take a prospect of the creatures for a discovery of God. 
The apostles drew arguments from the topics of nature, 
    when they discoursed with those that owned the Scripture (Rom. 1:19), 
    as well as when they treated with those that were ignorant of it, 
    as Acts 14:16, 17. 
And among the philosophers of Athens (Acts 17:27, 29), 
    such arguments the Holy Ghost in the apostles thought sufficient 
    to convince men of the existence, unity, spirituality, and patience of God. 
Such arguments had not been used by them and the prophets 
    from the visible things in the world to silence the Gentiles 
    with whom they dealt, 
    had not this truth, and, much more about God, 
    been demonstrable by natural reason: 
    they knew well enough that probable arguments 
    would not satisfy piercing and inquisitive minds.\footnote{Ibid.}

In Paul‘s account, the testimony of the creatures was without contradiction. 
God himself justifies this way of proceeding by his own example, 
    and remits Job to the consideration of the creatures, 
    to spell out something of his divine perfections.\footnote{Job 28:89, 40, \&c. 
        It is but one truth in philosophy and divinity
        that which is false in one, 
        cannot be true in another;
        truth, in what appearance soever,
        doth never contradict itself.}
And this is so convincing an argument of the existence of God, 
    that God never vouchsafed any miracle,
    or put forth any act of omnipotency, 
    besides what was present in the creatures, 
    for the satisfaction of the curiosity of any atheist, 
    or the evidencing of his being, 
    as he hath done for the evidencing those truths 
    which were not written in the book of nature, 
    or for the restoring a decayed worship, 
    or the protection or deliverance of his people. 
Those miracles in publishing the gospel,
    indeed, did demonstrate the existence of some supreme power; 
    but they were not seals designedly affixed for that, 
    but for the confirmation of that truth, 
    which was above the ken of purblind reason,
    and purely the birth of divine revelation. 
Yet what proves the truth of any spiritual doctrine, 
    proves also in that act the existence of the Divine author of it. 
The revelation always implies a revealer, 
    and that which manifests it to be a revelation, 
    manifests also the supreme Revealer of it. 
By the same light the sun manifests other things us, 
    it also manifests itself. 
But what miracles could rationally supposed to work on an atheist, 
    who is not drawn to a sense of the truth proclaimed aloud 
    by so many wonders of the creation? 
Let us now proceed to the demonstration of the atheist’s folly.

\section{The atheist's folly}
It is a folly to deny or doubt of a Sovereign Being, 
    incomprehensible in his nature, 
    infinite in his essence and perfections, 
    independent in his operations, 
    who hath given being to the whole frame 
    of sensible and intelligible creatures, 
    and governs them according to their several natures,
    by an inconceivable wisdom,
    who fills the heavens with the glory of his majesty, 
    and the earth with the influences of his goodness.

It is a folly inexcusable to renounce, in this case, 
    all appeal to universal consent, 
    and the joint assurances of the creatures.

\subsection{To deny the universal acknowledgment of all nations}
Reason I. ’Tis a folly to deny or doubt of that which hath been
    the acknowledged sentiment of all nations, 
    in all places and ages.
There is no nation but hath owned some kind of religion, 
    and, therefore, no nation but hath consented in the notion of 
    a Supreme Creator and Governor.

1. This hath been universal 
2. It hath been constant and uninterrupted. 
3. Natural and innate.

%% Begin 0041.
\subsubsection{In judgments and practices}
First, It hath been universally assented to by 
    the judgments and practices of all nations in the world.

1. No nation hath been exempt from it. 
    All histories of former and latter ages have not produced 
    any one nation but fell under the force of this truth. 
Though they have differed in their religions they have agreed in this truth; 
    here both heathen, Turk, Jew, and Christian, 
    centre without any contention. 
No quarrel was ever commenced upon this score; 
    though about other opinions wars have been sharp, 
    and enmities irreconcilable. 
The notion of the existence of a Deity was the same in all, 
    Indians as well as Britons, 
    Americans as well as Jews. 
It hath not been an opinion peculiar to this or that people, 
    to this or that sect of philosophers; 
    but hath been as universal as the reason whereby men are 
    differenced from other creatures,
    so that some have rather defined man by \emph{animal religiosum}, 
    than \emph{animal rationale}. 
’Tis so twisted with reason that a man cannot be accounted rational, 
    unless he own an object of religion; 
    therefore he that understands not this, 
    renounceth his humanity when he renounceth a Divinity. 
No instance can be given of any one people in the world that disclaimed it. 
It hath been owned by the wise and ignorant, 
    by the learned and stupid, 
    by those who had no other guide but the dimmest light of nature, 
    as well as those whose candles were snuffed by a more polite education, 
    and that without any solemn debate and contention. 
Though some philosophers have been known to change their opinions 
    in the concerns of nature, 
    yet none can be proved to have absolutelv changed their opinion 
    concerning the being of a God. 
One died for asserting one God; 
    none, in the former ages upon record, hath died for asserting no God. 
Go to the utmost bounds of America, 
    you may find people without some broken pieces of the law of nature, 
    but not without this signature and stamp upon them, 
    though they wanted commerce with other nations,
    except as savage as themselves, 
    in whom the light of nature was as it were sunk into the socket, 
    who are but one remove from brutes,
    who clothe not their bodies, cover not their shame, 
    yet were they as soon known to own a God, as they were known to be a people.
They were possessed with the notion of a Supreme Being, 
    the author of the world; 
    had an object of religious adoration; 
    put up prayers to the deity they owned for the good things they wanted, 
    and the diverting the evils they feared. 
No people so untamed where absolute perfect atheism had gained a footing. 
No one nation of the world known in the time of the Romans 
    that were without their ceremonies,
    whereby they signified their devotion to a deity. 
They had their places of worship, 
    where they made their vows, 
    presented their prayers, 
    offered their sacrifices, 
    and implored the assistance of what they thought to be a god; 
    and in then distresses run immediately,
    without any deliberation, to their gods: 
    so that the notion of a deity was as inward and settled in them 
    as their own souls, and, indeed, runs in the blood of mankind. 
The distempers of the understanding cannot utterly deface it; 
    you shall scarce find the most distracted bedlam, 
    in his raving fits, to deny a God, 
    though he may blaspheme, and fancy himself one.

2. Nor doth the idolatry and multiplicity of gods in the world weaken, %% Begin 0042.
    but confirm this universal consent. 
Whatsoever unworthy conceits men have had of God in all nations, 
    or whatsoever degrading representations they have made of him, 
    yet they all concur in this, 
    that there is a Supreme Power to be adored. 
Though one people worshipped the sun, others the fire,---and the Egyptians, 
    gods out of their rivers, gardens, and fields; 
    yet the notion of a Deity existent, 
    who created and governed the world, 
    and conferred daily benefits upon them, 
    was maintained by all, 
    though applied to the stars, 
    and in part to those sordid creatures. 
All the Dagons of the world establish this truth, 
    and fall down before it.
Had not the nations owned the being of a God, 
    they had never offered incense to an idol: 
    had there not been a deep impression of the existence of a Deity, 
    they had never exalted creatures below themselves 
    to the honor of altars: 
    men could not so easily have been deceived by forged deities, 
    if they had not had a notion of a real one. 
Their fondess to set up others in the place of God, 
    evidenced a natural knowledge that there was One 
    who had a right to be worshipped.
If there were not this sentiment of a Deity, 
    no man would ever have made an image of a piece of wood, 
    worshipped it, prayed to it, and said, 
    “ Deliver me, for thou art my God.”\footnote{Isaiah 44:17.} 
They applied a general notion to a particular image. 
The difference is in the manner, 
    and immediate object of worship, 
    not in the formal ground of worship.
The worship sprung from a true principle, 
    though it was not applied to a right object: 
    while the were rational creatures, 
    they could not deface the notion; 
    yet while they were corrupt creatures it was not difficult 
    to apply themselves to a wrong object from a true principle. 
A blind man knows he hath a way to go as well as one of the clearest sight; 
    but because of his blindness he may miss the way and stumble into a ditch. 
No man would be imposed upon to take a Bristol stone instead of a diamond, 
    if he did not know that there were such things as diamonds in the world: 
    nor any man spread forth his hands to an idol, 
    if he were altogether without the sense of a Deity. 
Whether it be a false or a true God men apply to, yet in both, 
    the natural sentiment of a God is evidenced;
    all their mistakes were grafts inserted in this stock, 
    since they would multiply gods rather than deny a Deity.

How should such a general submission be entered into by all the world, 
    so as to adore things of a base alloy,\footnote{Charron de la Sagesse, Liv. i. ch. 7p. 48, 44.}
    if the force of religion were not such, 
    that in any fashion a man would seek the satisfaction 
    of his natural instinct to some object of worship? 
This great diversity confirms this consent to be a good argument, 
    for it evidenceth it not to be a cheat, combination 
    or conspiracy to deceive, or a mutual intelligence, 
    but every one finds it in his climate, 
    yea in himself.
People would never have given the title of a God to men or brutes
    had there not been a pre-existing and unquestioned persuasion, 
    that there was such a being;---how else should 
    the notion of a God come into their minds?---the notion 
    that there is a God must be more ancient.\footnote{Gassend. Phys. § 1, lib. iv. c. 2. p. 291.}

3. Whatsoever disputes there have been in the world,
    this of the existence of God was never the subject of contention. 
All other things have been questioned.  %% Page 0043.
What jarrings were there among philosophers about natural things! 
    into how many parties were they split! 
    with what animosities did they maintain their several judgments! 
    but we hear of no solemn controversies about the existence of a Supreme Being: 
    this never met with any considerable contradiction: 
    no nation, that hath put other things to question, 
    would ever suffer this to be disparaged, 
    so much as by a public doubt. 
We find among the heathen contentions about the nature of God 
    and the number of gods, 
    some asserted an innumerable multitude of gods,
    some affirmed him to be subject to birth and death, 
    some affirmed the entire world was God; 
    others fancied him to be a circle of a bright fire; 
    others that he was a spirit diffused through the whole world:\footnote{Amyrant des Religion, p. 50}
    yet they unanimously concurred in this, 
    as the judgment of universal reason, 
    that there was such a sovereign Being: 
    and those that were skeptical in everything else, 
    and asserted that the greatest certainty was that there was nothing certain, 
    professed a certainty in this. 
The question was not whether there was a First Cause, 
    but what it was. 
It is much the same thing, 
    as the disputes about the nature and matter of the heavens, 
    the sun and planets, 
    though there be great diversity of judgments, 
    yet all agree that there are heavens, sun, planets; 
    so all the contentions among men about the nature of God,
    weaken not, but rather coniirm, that there is a God, 
    since there was never a public formal debate about his existence.\footnote{Gassend. Phys. § 1, lib iv. c. 2. p. 291.}
Those that have been ready to pull out one another’s eyes 
    for their dissent from their judgments, 
    sharply censured one another’s sentiments, 
    envied the births of one another’s wits, 
    always shook hands with an unanimous consent in this; 
    never censured one another for being of this persuasion, 
    never called it into question; 
    as what was never controverted among men professing Christianity, 
    but acknowledged by all, 
    though contending about other things, 
    has reason to be judged a certain truth belonging to the christian religion; 
    so what was never subjected to any controversy, 
    but acknowledged by the whole world, 
    the reason to be embraced as a truth without any doubt.

4. This universal consent is not prejudiced by some few dissenters.
History doth not reckon twenty professed atheists in all ages 
    in the compass of the whole world: 
    and we have not the name of an one absolute atheist 
    upon record in Scripture; 
    yet it is questioned whether any of them, 
    noted in history with that infamous name, 
    were downright deniers of the existence of God, 
    but rather because they disparaged the deities commonly worshipped 
    by the nations where they lived, 
    as being of a clearer reason to discern that those qualities,
    vulgarly attributed to their gods, 
    as lust and luxury, 
    wantonness and quarrels, 
    were unworthy of the nature of a god.\footnote{Gassend. Phys. § 1, lib iv. c. 7. p. 282.}
But suppose they were really what they termed to be, 
    what are they to the multitude of men 
    that have sprung out of the loins of Adam? 
    not so much as one grain of ashes 
    is to all that were ever turned into that form by any fires in your chimneys. 
And many more were not sufficient
    to weigh down the contrary consent of the whole world, 
    and bear down an universal impression. 
Should the laws of a country, 
    agreed universally to by the whole body of the people, 
    be accounted vain,
    because an hundred men of those millions disapprove of them, 
    when not their reason, 
    but their folly and base interest, 
    persuades them to dislike them and dispute against them? 
What if some men be blind?
    shall any conclude from thence that eyes are not natural to men?
    shall we say that the notion of the existence of God 
    is not natural to men, 
    because a very small number have been of a contrary opinion?
    shall a man in a dungeon, that never saw the sun, deny that there is a sun, 
    because one or two blind men tell him there is none, 
    when thousands assure him there is.\footnote{Gassend. ibid. p. 290.}
Why should then the exceptions of a few, not one to millions, 
    discredit that which is voted certainly true 
    by the joint consent of the world? 
Add this, too, that if those that are reported to be atheists 
    had had any considerable reason 
    to step aside from the common persuasion of the whole world 
    it is a wonder it met not with entertainment by great numbers of those, 
    who, by reason of their notorious wickedness and inward disquiets, 
    might reasonably be thought to wish in their hearts 
    that there were no God. 
It is strange if there were any reason on their side, 
    that in so long a space of time as hath run out from the creation of the world,
    there could not be engaged a considerable number 
    to frame a society by the profession of it. 
It hath died with the person that started it,
    and vanished as soon as it appeared.

To conclude this, 
    is it not folly for any man to deny or doubt of the being of a God, 
    to dissent from all mankind, and stand in contradiction to human nature? 
What is the general dictate of nature is a certain truth. 
It is impossible that nature can naturally and universally lie. 
And therefore those that ascribe all to nature, 
    and set it in the place of God, contradict themselves, 
    if they give not credit to it in that which it universally aiffirms. 
A general consent of all nations is to be esteemed as a law of nature.\footnote{Cicero.} 
Nature cannot plant in the minds of all men an assent to a falsity, 
    for then the laws of nature would be destructive to the reason and minds of men.
How is it possible, that a falsity should be a persuasion spread
    through all nations, 
    engraven upon the minds of all men, 
    men of the most towering, 
    and men of the most creeping understanding; 
    that they should consent to it in all places, 
    and in those places where the nations have not had any known commerce 
    with the rest of the known world? 
    a consent not settled by any law of man to constrain people to a belief of it: 
    and indeed it is impossible that any law of man can constrain 
    the belief of the mind. 
Would not he deservedly be accounted a fool, 
    that should deny that to be gold which hath been tried and examined 
    by a great number of knowing goldsmiths,
    and hath passed the test of all their touch-stones?
What excess folly would it be for him to deny it to be true gold, 
    if it had been tried by all that had skill in that metal 
    in all nations in the world!
%% Page 0045
\subsubsection{Constant and uninterrupted}
Secondly, It hath been a constant and uninterrupted consent. 
It hath been as ancient as the first age of the world; 
    no man is able to mention any time, 
    from the beginning of the world, 
    wherein this notion hath not been universally owned; 
    it is as old as mankind,
    and hath run along with the course of the sun, 
    nor can the date be fixed lower than that.

1. In all the changes of the world, this hath been maintained. 
In the overtumings of the government of states, 
    the alteration of modes of worship, 
    this hath stood unshaken. 
The reasons upon which it was founded were, in all revolutions of time, 
    accounted satisfactory and convincing, 
    nor could absolute atheism in the changes of any laws 
    ever gain the favor of any one body of people to be established by a law. 
When the honor of the heathen idols was laid in the dust, 
    this suffered no impair. 
The being of one God was more vigorously owned when 
    the unreasonableness of multiplicity of gods was manifest; 
    and grew taller by the detection of counterfeits. 
When other arts of the law of nature have been violated by some nations, 
    this hath maintained its standing. 
The long series of ages hath been so far from blotting it out, 
    that it hath more strongly confirmed it, 
    and maketh further progress in the confirmation of it. 
Time, which hath eaten out the strength of other things, 
    and blasted mere inventions, hath not been able to consume this. 
The discovery of all other impostures, 
    never made this by any society of men to be suspected as one. 
It will not be easy to name any imposture that hath walked 
    perpetually in the world without being discovered, 
    and whipped out by some nation or other. 
Falsities have never been so universally and constantly owned 
    without public control and question. 
And since the world hath detected many errors of the former age, 
    and learning been increased, this hath been so far from being dimmed,
    that it hath shone out clearer with the increase of natural knowledge,
    and received fresh and more vigorous conirmations.

2. The fears and anxieties in the consciences of men have given
    men suifficient occasion to root it out, 
    had it been possible for them to do it. 
If the notion of the existence of God had been possible 
    to have been dashed out of the minds of men, 
    they would have done it rather than have suffered 
    so many troubles in their souls upon the commission of sin; 
    since there did not want wickedness and wit 
    in so many corrupt ages to have attempted it and prospered in it, 
    had it been possible. 
How comes it therefore to pass, 
    that such a multitude of profligate persons that have been in the world 
    since the fall of man, 
    should not have rooted out this principle, 
    and dispossessed the minds of men of that which gave birth 
    to their tormenting fears?
How is it possible that all should agree together in a thing which created fear, 
    and an obligation against the interest of the flesh, 
    if it had been free for men to discharge themselves of it? 
    No man, as far as corrupt nature bears sway in him, 
    is willing to live controlled.

The first man would rather be a god himself than under one:\footnote{Gen. 3:5.}
    why should men continue this notion in them, 
    which shackled them in their vile inclinations, 
    if it had been in their power utterly to deface it? 
If it were an imposture, how comes it to pass, 
    that all the wicked ages of the world could never 
    discover that to be a cheat,
    which kept them in continual alarms? 
Men wanted not will to shake off such apprehensions; 
    as Adam, so all his posterity are desirous 
    to hide themselves from God upon the commission of sin,\footnote{Gen. 3:9.}
    and by the same reason they would hide God from their souls. %% page 0046.
What is the reason they could never attain their will and their wish 
    by all their endeavors? 
Could they possibly have satisfied themselves 
    that there were no God, 
    they had discarded their fears, 
    the disturbers of the repose of their lives, 
    and been unbridled in their pleasures. 
The wickedness of the world would never have preserved 
    that which was a perpetual molestation to it, 
    had it been possible to be razed out.

But since men under the turmoils and lashes of their own consciences 
    could never bring their hearts to a settled dissent 
    from this truth, it evidenceth, 
    that as it took its birth at the beginning of the world, 
    it cannot expire, no not in the ashes of it, 
    nor in anything but the reduction of the soul 
    to that nothing from whence it sprung.
This conception is so perpetual, 
    that the nature of the soul must be dissolved before it be rooted out, 
    nor can it be extinct while the soul endures.

3. Let it be considered also by us that own the Scripture, 
    that the devil deems it impossible to root out this sentiment. 
It seems to be so perpetually fixed, 
    that the devil did not think fit to tempt man to 
    the denial of the existence of a Deity, 
    but persuaded him to believe he might ascend to that deity 
    and become a god himself; 
    Gen. 3:1, ``Hath God said?'' and he there owns him (ver. 5), 
    ``Ye shall become as gods.'' 
He owns God in the question he asks the woman,
    and persuades our first parents to be gods themselves. 
And in all stories, both ancient and modern, 
    the devil was never able to tincture men’s minds 
    with a professed denial of the Deity, 
    which would have opened a door to a world of more wickedness 
    than hath been acted,
    and took away the bar to the breaking out of that evil, 
    which is naturally in the hearts of men, 
    to the greater prejudice of human societies. 
He wanted not malice to raze out all the notions of God, but power: 
    he knew it was impossible to effect it, 
    and therefore in vain to attempt it. 
He set up himself in several places of the ignorant world as a god, 
    but never was able to overthrow the opinion of the being of a God. 
The impressions of a Deity were so strong as not to be struck out 
    by the malice and power of hell.

What a folly is it then in any to contradict or doubt of this truth,
    which all the periods of time have not been able to wear out; 
    which all the wars and quarrels of men with their own consciences 
    have not been able to destroy; 
    which ignorance and debauchery, its two greatest enemies, cannot weaken; 
    which all the falsehoods and errors which have reigned 
    in one or other part of the world, 
    have not been able to banish; 
    which lives in the consents of men in spite of all their wishes 
    to the contrary, 
    and hath grown stronger, and shone clearer, 
    by the improvements of natural reason!

\subsubsection{Natural and innate}
Thirdly, Natural and innate; which pleads strongly for the perpetuity of it. 
It is natural, though some think it not a principle in the heart of man;\footnote{Pink. Eph. 6, p. 10, 11.}
    but it is so natural that every man is born 
    with restless instinct to be of some kind of religion or other, 
    which implies some object of religion. 
The impression of a Deity is as common as reason, 
    and of the same age with reason.\footnote{King on Jonah, p. 16.}
It is a relic of knowledge after the fall of Adam, 
    like fire under ashes, 
    which sparkles as soon
    %% page 0047.
    as ever the heap of ashes is opened. 
A notion sealed up in the soul of every man;\footnote{Amyrant des Religions, p.6-9}
    else how could those people who were unknown to one another, 
    separate by seas and mounts, 
    differing in various customs and manner of living, 
    had no mutual intelligence one with another, 
    light upon this as a common sentiment, 
    if they had not been guided by one uniform reason in all their minds, 
    by one nature common to them all: 
    though their climates be different, 
    their tempers and constitutions various, 
    their imaginations in some things as distant from one another 
    as heaven is from earth, 
    the ceremonies of their religion not all of the same kind; 
    yet wherever you find human nature, 
    you find this settled persuasion. 
So that the notion of a God seems to be twisted with the nature of man, 
    and is the first natural branch of common reason, 
    or upon either the first inspection of a man into himself 
    and his own state and constitution, 
    or upon the first sight of any external visible object. 
Nature within man, and nature without man, 
    agree upon the first meeting together to form this sentiment, 
    that there is a God. 
It is as natural as anything we call a common principle. 
One thing which is called a common principle and natural is, 
    that the whole is greater than the parts. 
If this be not born with us, 
    yet the exercise of reason essential to man settles it as a certain maxim; 
    upon the dividing anything into several parts,
    he finds every part less than when they were altogether. 
By the same exercise of reason, 
    we cannot cast our eyes upon anything in the world, 
    or exercise our understandings upon ourselves, 
    but we must presently imagine, 
    there was some cause of those things, 
    some cause of myself and my own being; 
    so that this truth is as natural
    to man as anything he can call most natural or a common principle.

It must be confessed by all, 
    that there is a law of nature writ upon the hearts of men, 
    which will direct them to commendable actions,
    if they will attend to the writing in their own consciences. 
This law cannot be considered without the notice of a Lawgiver. 
For it is but a natural and obvious conclusion, 
    that some superior hand engrafted those principles in man, 
    since he finds something in him twitching him upon 
    the pursuit of uncomely actions, 
    though his heart be mightily inclined to them; 
    man knows he never planted this principle of reluctancy in his own soul; 
    he can never be the cause of that which he cannot be friends with. 
If he were the cause of it, why doth he not rid himself of it? 
No man would endure a thing that doth frequently molest and disquiet him, 
    if he could cashier it. 
It is therefore sown in man by some hand more powerful than man,
    which riseth so high, and is rooted so strong, 
    that all the force that man can use cannot pull it up. 
If therefore this principle be natural in man, 
    and the law of nature be natural, 
    the notion of a Lawgiver must be as natural, 
    as the notion of a printer, 
    or that there is a printer, 
    is obvious upon the sight of a stamp impressed. 
After this the multitude of effects in the world step in 
    to strengthen this beam of natural light, 
    and the direct conclusion from thence is, 
    that that power which made those outward objects, 
    implanted this inward principle. 
This is sown in us, born with us, 
    and sprouts up with our growth, or as one saith; 
    it is like letters carved upon the bark of a %% page 0048.
    young plant, 
    which grows up together with us, 
    and the longer it grows the letters are more legible.\footnote{Charleton}

This is the ground of this universal consent, 
    and why it may well be termed natural. 
This will more evidently appear to be natural, because,

1. This consent could not be by mere tradition. 
2. Nor be any mutual intelligence of governors to keep people in awe, 
    which are two things the atheist pleads; 
    the first hath no strong foundation,
    and that other is as absurd and foolish 
    as it is wicked and abominable. 
3. Nor was it fear first introduced it.

First, It could not be by mere tradition. Many things indeed are
entertained by posterity which their ancestors delivered to them, and
that out of a common reverence to their forefathers, and an opinion
that they had a better prospect of things than the increase of the
corruption of succeeding ages would permit them to have. But if
this be a tradition handed from our ancestors, they also must receive
it from theirs; we must then ascend to the first man, we cannot
else escape a confounding ourselves with running into infinite. Was
it then the only tradition he left to them? Is it not probable he
acquainted them with other things in conjunction with this, the
nature of God, the way to worship him, the manner of the world’s
existence, his own state? We may reasonably suppose him to have
a good stock of knowledge; what is become of it? It cannot be
supposed, that the first man should acquaint his posterity with an
object of worship, and leave them ignorant of a mode of worship
and of the end of worship. We find in Scripture his immediate
posterity did the first in sacrifices, and without doubt they were not
ignorant of the other: how come men to be so uncertain in all other
things and so confident of this, if it were only a tradition? How
did hates and irreconcilable questions start up concerning other
things, and this remain untouched, but by a small number? 
Whatsoever the tradition the first man left besides this, is lost, and no way
recoverable, but by the revelation God hath made in his Word.
How comes it to pass this of a God is longer lived than all the rest
which we may suppose man left to his immediate descendants? How
come men to retain the one and forget the other? What was the
reason this survived the ruin of the rest, and surmounted the 
uncertainties into which the other sunk? Was it likely it should be
handed down alone without other attendants on it at first? Why
did it not expire among the Americans, who have lost the account
of their own descent, and the stock from whence they sprung, and
cannot reckon above eight hundred or a thousand years at most?
Why was not the manner of the worship of a God transmitted as
well as that of his existence? How came men to dissent in their
opinions concerning his nature, whether he was corporeal or 
incorporeal, finite or infinite, omnipresent or limited? Why were not men
so negligent to transmit this of his existence as that of his nature?
No reason can be rendered for the security of this above the other,
but that there is so clear a tincture of a Deity upon the minds of
men, such traces and shadows of him in the creatures, such indelible
instincts within, and invincible arguments without to keep up this %% 0049
universal consent. The characters are so deep that they cannot
possibly be rased out, which would have been one time or other, in
one nation or other, had it depended only upon tradition, since one
age shakes off frequently the sentiments of the former. I cannot
think of above one which may be called a tradition, which indeed
was kept up among all nations, viz. sacrifices, which could not be
natural but instituted. What ground could they have in nature, to
imagine that the blood of beasts could expiate and wash off the guilt
and stains of a rational creature? Yet they had in all places (but
among the Jews, and some of them only) lost the knowledge of the
reason and end of the institution, which the Scripture acquaints us
was to typify and signify the redemption by the Promised Seed.
This tradition hath been superannuated and laid aside in most parts
of the world, while this notion of the existence of a God hath stood
firm. But suppose it were a tradition, was it likely to be a mere
intention and figment of the first man? Had there been no reason for
it, this posterity would soon have found out the weakness of its foundation. 
What advantage had it been to him to transmit so great a
falsehood to kindle the fears or raise the hopes of his posterity, if
there were no God? It cannot be supposed he should be so void
of that natural affection men in all ages bear to their descendants,
as so grossly to deceive them, and be so contrary to the simplicity
and plainness which appears in all things nearest their original.

Secondly, Neither was it by any mutual intelligence of governors
among themselves to keep people in subjection to them. If it were
a political design at first, it seems it met with the general nature of
mankind very ready to give it entertainment.

1. It is unaccountable how this should come to pass. It must be
either by a joint assembly of them, or a mutual correspondence. If
by an assembly, who were the persons? Let the name of any one
be mentioned. When was the time? Where was the place of this
appearance? By what authority did they meet together? Who
made the first motion, and first started this great principle of policy?
By what means could they assemble from such distant parts of the
world? Human histories are utterly silent in it, and the Scripture,
the ancientest history, gives an account of the attempt of Babel, but
not a word of any design of this nature. What mutual correspondence 
could such have, whose interests are for the most art different, 
and their designs contrary to one another? How could they, who
were divided by such vast seas, have this mutual converse? How
could those who were different in their customs and manners, agree
so unanimously together in one thing to gull the people? If there
had been such a correspondence between the governors of all nations,
what is the reason some nations should be unknown to the world
till of late times? How could the business be so secretly managed,
as not to take vent, and issue in a discovery to the world? Can
reason suppose so many in a joint conspiracy, and no man`s 
conscience in his life under sharp attractions, or on his death-bed, when
conscience is most awakened, constrain him to reveal openly the
cheat that beguiled the world? How came they to be so unanimous
in this notion, and to differ in their rites almost in every country? %% 0050.
why could they not agree in one mode of worship throughout all
the world, as well as in this universal notion? If there were not a
mutual intelligence, it cannot be conceived how in every nation such
a state-engineer should rise up with the same trick to keep people in
awe. What is the reason We cannot find any law in any one nation
to constrain men to the belief of the existence of a God, since politic
stratagems have been often fortified by laws? Besides, such men
make use of principles received to effect their contrivances, and are
not so impolitic as to build designs upon principles that have no
foundation in nature. Some heathen lawgivers have pretended a
converse with their gods, to make their laws be received by the
people with a greater veneration, and fix with stronger obligation
the observance and perpetuity of them; but this was not the 
introducing a new principle; but the supposition of an old received notion,
that there was a God, and an application of that principle to their
present design. The pretence had been vain had not the notion of a
God been ingrafted. Politicians are so little possessed with a 
reverence of God that the first might one in the Scripture (which may
reasonably gain with the atheist the credit of the ancientest history
in the world), is represented without any fear of God.\footnote{Genesis 10:9, 
    ``Nimrod was a mighty hunter before the Lord.}
An invader and oppressor of his neighbors, and reputed the introducer of a new
worship, and being the first that built cities after the flood (as Cain
was the first builder of them before the flood), built also idolatry
with them, and erected a new worship, and was so far from 
strengthening that notion the people had of God, that he endeavored to 
corrupt it. The first idolatry in common histories being noted to
proceed from that part of the world; the ancientest idol being at
Babylon, and supposed to be first invented by this person: whence,
by the way, perhaps Rome is in the Revelations called Babylon, with
respect to that similitude of their saint-worship, to the idolatry first
set up in that place.\footnote{Or if we understand it as some think, 
    that he defended his invasions under a pretext of the preserving religion,
    it assures us that there was a notion of an object of religion before, 
    since no religion can be without an object of worship.}
’Tis evident politicians have often changed the worship of a nation, 
but it is not upon record that the first thoughts of an object of worship 
ever entered into the minds of people by any trick of theirs.

But to return to the present argument, the being of a God is
owned by some nations that have scarce any form of policy among
them. 'Tis as wonderful how any wit should hit upon such an
invention, as it is absurd to ascribe it to any human device, if there
were not prevailing arguments to constrain the consent. Besides,
how is it possible they should deceive themselves? What is the
reason the greatest politicians have their fears of a Deity upon their
unjust practices, as well as other men they intend to befool? How
many of them have had forlorn consciences upon a death-bed, upon
the consideration of a God to answer an account to in another world?
Is it credible they should be frighted by that wherewith they knew
they beguiled others? No man satisfying his pleasures would im-

\end{document}
