This psalm is a description of the deplorable corruption by nature
of every son of Adam, since the withering of that common root.
Some restrain it to the Gentiles, as a wilderness full of briers and
thorns, as not concerning the Jews, the garden of God, planted by
his grace, and watered by the dew of heaven. But the apostle, the
best interpreter, rectifies this in extending it by name to Jews, as
well as Gentiles, (Rom. 3:9) ``We have before proved both Jews
and Gentiles, that they are all under sin;'' and (ver. 10-12) cites
part of this psalm and other passages of scripture for the further evidence
of it, concluding by Jews and Gentiles, every person in the
world naturally in this state of corruption.

The psalmist first declares the corruption of the faculties of the
soul, \emph{The fool hath said in his heart}; secondly, the streams issuing
from thence, \emph{they are corrupt}, \&c.: the first in atheistical principles,
the other in unworthy practice; and lays all the evil, tyranny, lust,
and persecutions by men, (as if the world were only for their sake)
upon the neglects of God, and the atheism cherished in their hearts.

\emph{The fool}, a term in scripture signifying a wicked man, used also
by the heathen philosophers to signify a vicious person, HEBREW as coming
from HEBREW signifies the extinction of life in men, animals, and
plants; so the word HEBREW is taken, a plant that hath lost all that juice
that made it lovely and useful.\footnote{Isaiah 40:7. HEBREW ``the flower fadeth.'', 
Isaiah 28:1} So a fool is one that hath lost
his wisdom, and right notion of God and divine things which were
communicated to man by creation; one dead in sin, yet one not
so much void of rational faculties as of grace in those faculties, not
one that wants reason, but abuses his reason. In Scripture the word
signifies foolish.\footnote{Mais HEBREW and HEBREW put together. Deut. 32:6 ``O foolish people and unwise.''} 

\emph{Said in his heart}; that is, he thinks, or he doubts, or he wishes.
The thoughts of the heart are in the nature of words to God, though
not to men. It is used in the like case of the atheistical person,
(Ps. 10:11, 13) ``He hath said in his heart, God hath forgotten; he
hath said in his heart, Thou wilt not require it.'' He doth not form
a syllogism, as Calvin speaks, that there is no God: he dares not
